\documentclass{beamer}
\usepackage{graphicx}
\usepackage[pdf]{graphviz}
\usetheme{Copenhagen}
%Information to be included in the title page:
\author{Markus Schoetz}
\title{Twin Testing the Themis BFT Framework}
\institute{C4 - FAU Erlangen}
\date{2025}

\begin{document}

\frame{\titlepage}

\begin{frame}
\frametitle{Inhalt}
\tableofcontents
\end{frame}

\section{Ziel}
\begin{frame}
\frametitle{Ziel der Arbeit}
\begin{itemize}
 \item Was: Bachelorarbeit zum Implementieren von Twins in Themis
 \item Wie:
 \begin{itemize}
  \item Ausarbeiten eines Szenarios
  \item Umsetzen eines Szenario Generator
  \item Umsetzen eines Szenario Executor
 \end{itemize}
 \item Wozu: Abhärten von Themis
\end{itemize}
\end{frame}

\section{Twins}
\begin{frame}
\frametitle{Twins}
 \begin{itemize}
  \item Test Framework für BFT Algorithmen
  \item Generiert systematisch Test Szenarien (Generator)
  \item Simulation: Ausführen dieser Szenarien
  \item Validierung des Zustands (evtl. Zwischenzustände)
  % \begin{itemize}
  %  \item 
  % \end{itemize}
 \end{itemize}

% behaviors: (i) leader equivocation, (ii) double voting, and (iii)
% losing internal state such as forgetting \end{itemize}
\end{frame}

\subsection{Szenario}
\subsubsection{Partition}
\begin{frame}
\frametitle{Partition}
 \digraph{abc}{
   rankdir=LR;
   a -> b -> c;
 }
\end{frame}
\subsubsection{Szenario genauer}
\begin{frame}
 \frametitle{Szenario}
 \begin{itemize}
  \item Anfangszustand:
  \begin{itemize}
   \item Liste der Nodes mit Twin $n_4$: $\{n_1,n_2,n_3,n_{4,1},n_{4,2}\}$

   \item Initial Zustand der Nodes
   \item Liste der Clients: $\{c_1,c_2\}$
  \end{itemize}
  \item Ablauf:
  \begin{itemize}
   \item Auslöser für Zustandsübergang (Protokoll Schritt)
   \item Liste von Partitionen
   \item Liste von Client Nachrichten
  \end{itemize}

 \end{itemize}
\end{frame}

\subsection{Executor}
\begin{frame}
\frametitle{Executor}
"Führt Szenario aus"
 \begin{itemize}
  \item Monitior des Zustands (Network Layer Injection in Tokio)
  \item Konfiguriert Network Layer Injection
  \item Simuliert Client Anfragen aus Szenario
  \item Validiert Zustand (Safety / Liveness)
 \end{itemize}
\end{frame}

\section{Zutun}
\begin{frame}
\frametitle{Nächste Schritte}
Arbeite gerade an Szenario Generator, dazu:
 \begin{itemize}
  \item Genauer definieren von Szenario Datenstruktur
  \item Überlegungen zum Einschränken der Szenariomenge
 \end{itemize}
\end{frame}
\end{document}
